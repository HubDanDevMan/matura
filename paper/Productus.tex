\section{Product}

The main body of our project is a series of demo programs that work independently.
These demos can be separated into two sections the core features and gadgets.
As the name implies the core features play an integral part in making the operating
system work the core features are as follows:


\begin{itemize}

\item \textbf{Bootloader} \\
The bootloader loads the kernel into memory.
\item \textbf{Keyboard driver} \\
The keyboard driver handles the input of the keyboard with IO ports. This works
by it sending a signal to a certain IO port and then checking for a return signal.
Then the signal gets translated via a lookup table to ascii characters.
\item \textbf{Shell} \\
The shell is a terminal that allows for user input to activate commands on a command
line. It does this by keyinput from the keyboard driver and then saves said input
into a buffer. The string that is in the buffer can then be used to do operations with
it. This is done by the string comparison function from the library with common functions.
\item \textbf{Filesystem} \\
The filesystem organized and structures files. The filesystem is not hierarchical meaning
that it does not support multiple directories. It has both file read and file write functions
and it uses a superblock for file organization.
\item \textbf{CPU identification} \\
The CPU identification gives information on the current CPU used by the operating system.
This is done via a interrupt and gives information such as its family, wheather longmode is
supported, the manufacturer, what brand it is and the modell.
\item \textbf{Library with common fuctions} \\
The library with common function is a collections of functions that are often used in different
programs. The library contains following functions: Format to hex a function that converts decimal
numbers to hexadecimal numbers. Print buffer functions that prints out a buffer onto the screen. 
Shutdown function which shuts down the current running program. Clear screen funtion which wipes
the entire screen blank. And lastly the get string length function which as its name implies returns
the length of a string.
\item \textbf{Interrupt handlers} \\
The interrupt handler creates exceptions for common errors such as division by zero, double fault
and invalid opcode errors.


\end{itemize}

And of course there are also the gadgets:

\begin{itemize}

\item Painting Program \\
In the painting program you can draw with four-bit colors. It has both a curser and an incdicator 
which shows you the current selected color. There is only one brush size at the moment and the
shape of the brush is a square

\item Hangman \\
The hangman program is a game of hangman in which a random word is selected from a list. Then the
user has to guess the word by giving the program a letter then the program either reveals the guessed
letter or deducts from your tries. The game ends when either the number of tries reaches zero or the
word is completely guessed.
\item Texteditor \\
The texteditor program allows you to write texts via keyboard input and navigate the written text to
correct mistakes.

\end{itemize}



There are also some more minor gadgets not worth mentioning they mostly consist of minor function that 
were needed for specific tasks. The entire project is licenced under the GNU general public licence version 
3.0. This licence requires all modification and usage of our project to be under the same licence and 
require authors to revealthe full source code of the licenced medium. The entirety of the project is located
in a github repository under the following link:

https://github.com/HubDanDevMan/matura.git

And the full licence text can be found under:

https://github.com/HubDanDevMan/matura/blob/master/LICENSE
