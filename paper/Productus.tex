\section{Product}

The main body of our project consists of a series of demo programs that work independently.
These demos can be separated into the two sections core components and gadget programs.
As the name implies, the core components play an integral part in making the operating
system work. They are as follows:


\begin{itemize}

\item \textbf{Bootloader} \\
This component is responsible for loading the kernel into memory. The kernel in FlamingOS is loosely
comprised of the remaining components listed below.
\item \textbf{Keyboard driver} \\
The keyboard driver handles the input of the keyboard through \textbf{I/O} ports. The driver 
sends a signal to a certain \textbf{I/O} port and then checks for a response containing a key
scancode. Then the response of the keyboard gets translated via a lookup table to ascii characters
or special keycodes.
\item \textbf{Shell} \\
The shell is a terminal that allows the user to send commands in the form of text via a command
line. It does this by requesting key input from the keyboard driver and saving said input
into a buffer. The command in the buffer can then be evaluated and appropriately executed.
\item \textbf{Filesystem} \\
The filesystem organizes and structures files. The filesystem is not hierarchical, meaning
that it does not support multiple directories. It provides functions for reading and writing
files and it uses a superblock to organise the files.
\item \textbf{CPU identification} \\
The CPU identification gives information on the CPU used by the computer. This is done via the 
CPUID instruction and gives information such as its manufacturer, brand, CPU family, model and
whether it is a 64 bit processor.
\item \textbf{Library} \\
The library is a collection of functions that are used across various programs. The instructions
on the usage of these functions can be found in the file \texttt{library.asm} which is located
in the \texttt{demos} directory. The library contains the following functions: 
\begin{itemize}
\item \texttt{formatHex}: Formats a numeric value to the corresponding hexadecimal ascii representation.
\item \texttt{printBuff}: Print buffer functions that prints out a buffer onto the screen. 
\item \texttt{shutdown}: Shuts down the PC. 
\item \texttt{clear\_screen}: Wipes the entire screen blank.
\item \texttt{getStringLength}: Returns the length of a string.
\end{itemize}

\item \textbf{Interrupt handlers} \\
The interrupt handler takes care of exceptions such as division by zero, double fault
and invalid opcode errors.


\end{itemize}

And of course there are also \textit{gadget programs}:

\begin{itemize}

\item Painting Program \\
In the painting program the user can draw squares with four-bit colors. It has both a cursor and an indicator 
that displays the currently selected color. There is only one brush size at the moment and the
shape of the brush is a square.

\item Hangman \\
The program hangman is a game in which the player has to guess a word that was randomly chosen from a list. The program 
reveals whether the player chose a correct character after every keyboard input. If the player guessed correctly
the game continues, but the player gets one strike if a character that is not in the randomly selected word 
was chosen. The game ends when either the player guesses all the characters in the word or three strikes have
been accumulated.

\item Texteditor \\
The texteditor program allows the user to write and delete text via keyboard input and navigate through the written 
text to correct mistakes.

\end{itemize}



There are also some other minor gadget programs not worth mentioning brcause they mostly consist of minor functions 
that were needed for specific tasks. The entire project is licenced under the GNU general public licence version 
3.0. This licence requires all modification and usage of our project to be under the same licence and 
require authors to reveal the full source code of the licenced medium. The entirety of the project is located
in a GitHub repository under the following link:

\url{https://github.com/HubDanDevMan/matura.git}

And the full licence text can be found under:

\url{https://github.com/HubDanDevMan/matura/blob/master/LICENSE}
