\section{Workprocess}

\subsection{Workflow}

Our first step was deciding what our project should look like. We knew that we wanted to do something related to computer science, but we were still uncertain of the details. 
Rather spontaneously did we decide to make an operating system. The exact content of our matura project was decided after we talked to several different teachers. We 
appreciated their oppinions and we were warned of the adversities we would have to face if we chose to go through with it. The first challenge was to find find a teacher 
willing to oversee our project because we decided to write the operating system in 16-bit assembly. None of the teachers were familiar enough with assembly language, since
it is an old and nowadays rarely used programming language. Nonetheless, we embraced the challenge after we talked to Dr. Günther Palfinger. He was kind enough to accept our 
request and oversee our project. After a few weeks Dr. Palfinger reviewed the contract we set up where we decided on the terms and conditions. The second challenge was 
getting to know the assembly language. We started the research on the programing language and computer science in general even before we finished the contract. It was rather
tough in the beginning because assembly is not comparable to high-level languages like python which we picked up in school. Assembly language is much more intricate considering 
that it is closer to machine language which is the language understood by a computer. We had to invest approximately two months into research to comprehend the basics
of the language. Afterwards we each chose a task and started researching these specific topics individually. We listed the websites we used under the chapter "Sources".

\subsection{Tools}

Before we began programming we installed a virtual machine that 
allowed us to run Linux even though our host operating systems were either Windows or macOS. By using a virtual machine we protected our hardware from our own mistakes. The 
command line is a powerful tool that can, if used incorrectly, break your software and even hardware. But thanks to the virual machine we were able to isolate our host 
operating system from our working environment which was especially important as we were new to using the command line. Furthermore Linux is commonly used by specialists 
meaning that there are a lot of guides and tutorials to help beginners. On top of that there are more and usually better, convenient, free and open-source applications. We 
then installed the Netwide Assembler or NASM for short which, as the name suggest, is an assembler. It translates the assembly source code into machine language
so that the hardware can read and execute the program. Additionally we installed QEMU, a software that allowed us to emulate an entire computer system. QEMU enabled us to 
test our programs quickly and efficiently. To write the chapters and code we used neo vim which is a text editor focused on extensibility and usability. 
It can be an extremly powerful text editor if the user is experienced. While neo vim takes some getting used to it also provides the user with convenient shortcuts. 
Sometimes our code was faulty and we needed to debug it. For convience's sake we used an online debugger called GDB. This website provided us with quick and easy access to 
a reliable debugger. In order to save time and for simplicity and consitency we also used the make utility. We created a Makefile that contains instructions for building, 
emulating and debuging software. The command make executes the Makefile in the current directory and is followed by the name of the target(s). In our project the Makefile 
launches NASM, translating the source code, and then QEMU, emulating the operating system according to the parameters defined in the Makefile. We shared the Makefiles and 
everything else we made with git. The aforementioned is a distributed version control system that allowed us to manage our project quickly and efficiently. With git one can 
upload files to and download files from a repository. We used GitHub to store our files on a cloud so that we were able to work together on this project. Finally, we used Pandoc,
a universal document converter to convert our written text for the matura from \LaTeX  into a pdf. All of the previously mentioned programs are free and open-source.

\subsection{Workflow}

As soon as we installed the necessary software and got a grasp of the basics of assembly language we started programming. We chose easier tasks in the beginning so that
we could warm up to this rather complicated language. Whenever we were done writing a program we began debugging it. Some bugs cost us a great amount of time which 
was often very frustrating as we lost up to three weeks because of a single bug. Fortunately, we never lost hope and always continued our research. A few times we 
had to rewrite our code because we were either stuck or unhappy with the structure. This helped a lot because we had already gained new insights by the time we started rewriting 
the program. After a while the tasks we worked on got more complicated and the research had to be more extensive. When we were not in the mood to work on the programs or do 
research we wrote chapters explaining the most important parts of an operating system. Besides that, we contributed to the work journal every week to keep Dr. Palfinger up 
to date. In the final stages of our matura project we finished writing the chapters and debugging our programs. 
