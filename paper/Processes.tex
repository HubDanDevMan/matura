\section{Processes}

A process is defined as an instance of a computer program that is currently being executed.
The operating system is ultimately in control of all the processes and 
processes can be further categorized into what is called a \textit{process life cycle}. The process life cycle is
a categorization of the different states a process or program can be in. A process life cycle can vary
from one operating system to another and the names are also not standardized. The aforementioned 
categorisation consists of 5 different stages:

\begin{itemize}
\item Start: \\
	The original state in which a process is first initiated in. The kernel will assign a \textit{process identifier (PID)},
	usually a number. The PID is used to differnetiate between multiple processes. In this stage, the kernel will load the
	process instructions and data into memory.
\item Ready: \\
.	Processes will enter this state after the start state. The process is fully loaded in memory and is waiting for its \textit{time slice}.
		A time slice is a moment of usually a few microseconds where the process is actualy being run on the CPU. The \texttt{READY} state
		is also assigned to processes that already reached the running state but have not yet finished.
\item Running:
In this state the process has been assigned the CPU by the OS scheduler and executes its instructions. The process is currently in his time slice
and will stay there until it gets interrupted. The kernel will program a timer, usually a coprocessor, that counts down from a value defined by
		the kernel and will send an interrupt request to the kernel when the timer reaches 0. The kernel will run the timer interrupt handler
		that reconfigures the timer and change the process. Changing the process is also called \textit{context switch} and it involves
		saving the values in the registers, storing them in a dedicated area in RAM and changing the processes state from running to ready.
		The process that is replacing the previous process will have its registers restored from RAM and its state changed to \texttt{RUNNING}.
\item Waiting: \\
This is the stage in which the process is waiting for a resource such as opening a file. A file must be loaded from disk and this can take a bit of time.
During the stage of waiting, the process is unable to do other things and there is no point in the process occupying the CPU. This state is similar
to the \texttt{READY} state but the deciding factor in which the process will resume execution is not a signal by the timer but rather another hardware
signal, in the case of waiting for a file this signal is an interrupt from the hard disk that the content has been loaded into memory and ready. When
the kernel notices the interrupt of the storage device, it will check which process requested the file or another resource and put the process back into
the \texttt{RUNNING} state. Other examples of waiting for a resource are key input, mouse input and an internet packet.
\item Exit/Terminate: \\
This is the last state a process enters when it has finished with its execution. The processes instruction and data are then removed
from main memory. The operating system will free potential resources that are held by a process such as open files or allocated memory on the
heap\footnote{Processes, from: Medium, \today \\ https://medium.com/@imdadahad/a-quick-introduction-to-processes-in-computer-science-271f01c780da}.

\end{itemize}

\subsection{Binary executables}

Executables are files containing instructions and data for computers. When these files are \textit{run}, they
become procesess. Running a executable means that the contents of the file are loaded into memory. Executables
are associated with a format that describes how exactly the contents must be loaded. The loading itself
is done by the \textit{program loader}. It is a crucial component in operating systems. The program loader
maps the individual sections, primarily \texttt{.text} and \texttt{.data}, of an executable into memory according to the layout and position
specified in the executable. The format contains information about the executable itself, often also information on
how it was compiled and for what platform it is compiled for. An ELF-executable for Linux might not work on OpenBSD,
even though both are unix-like operating systems. They differ because an OS defines a \textit{calling convention} that
specifies how functions are called and how parameters are passed to functions. If an executable calls a library function on
an OS it wasn't compiled for, the parameters can get mixed up. This is a major disadvantage of binary executables. However,
they are the fastest type of programs.


\subsubsection{Scripts}

There is also a different type of executables, namely \textit{scripts}. Scripts are written in a human readable
scripting language. They rely on software to interprete the instructions at run time in contrast of
hardware (such as the CPU). The interpreting software is a program that contains instructions for
the CPU. Windows uses the \textbf{.bat} or \textbf{.cmd} extension for scripts written for the interpreter \texttt{cmd.exe} and for
the newer PowerShell scripts with extension \textbf{.ps1} it will use the interpreter called \texttt{powershell.exe}.
As long as a computer contains the correct interpreter for a script, the computer can run it effortlessly. This comes at
the cost of slower execution speeds.

\subsubsection{Executables on unix-like systems}

Unix-like operating systems differ greatly from Windows NT ones. They rarely rely on extensions to
identify executables but rather *file signatures*. Binary executables unix systems with the exception of MacOS
contain the \texttt{\textbackslash x7fELF} file signature and are in the ELF-format. A special type of file signature can be found on
unix scripts. Even though they are made of plain text characters, the author of the file creates the
signature by him or herself. In scripts for Unix-like operating systems, the format is as follows: \\
\texttt{\#!/path/to/the/script/interpreter -parameters\textbackslash n} followed by instructions in the scripting language that can
be interpreted by the interpreter specified in the path. The \textit{shebang} (\textbf{\#!}) is the script signature
and tells the kernel that the program is not in a binary format such as Mach-O (MacOS Darwin) or ELF (Linux et al.),
the unix-like counterparts to Windows \textbf{.exe} (PE). When a script executable is started,
the kernel will first invoke the interpreter, which is a binary executable and pass the name of the script
to the interpreter.


\subsection{Threads}

Threads are the most basic unit of CPU utilisation, meaning that this is the smallest amount that the workload of a 
process can be split. Threads consist of a stack, a set of registers and a program counter. Usually a process
is tied to one thread, meaning that the CPU does not split the workload over multiple threads. However, in
modern programming, multiple threads are used by one process to perform different tasks independent of each other
simultaneously. This is very useful because this means that a task in a process will not block 
other task. For example, a web browser can check for user input with one thread, load images with a second thread,
check for grammar errors on a third thread and make backups on a fourth thread. Each thread has its own set of registers
and a stack but all threads share the same files, code and data. Modern CPUs have multiple cores meaning that they have
multiple processing units that allow multiple threads to do parallel processing because each thread is strictly
related to one processing unit\footnote{ Multithreading on multiple cores, from: Askingalot, retrieved \today\\ \url{https://askinglot.com/does-multithreading-use-multiple-cores} }. This type of programming that uses threads to complete mutliple tasks
in parallel is reffered to as multi-threading and this benefits four main categories:

\begin{itemize}
\item Sclalability: \\

Multi-threading allows programmers to utilize multiple CPU cores for a single process as opposed to single
thread processes which can only utilize a single CPU core. This scales well on servers that have processors
with up to 32 cores.

\item Responsiveness: \\

When threads are still occupied with tasks in the process, multi-threading allows for another thread
to still check for input and thus allowes rapid responce from the user. Threads can by started on an
		\textit{as needed} basis, meaning if a process notices that it is very busy, it can start
		a new thread to increase responsiveness.

\item Execution speed: \\

Multi-threaded processes can much faster than single thread processes. Managing and creating threads allows for
		faster completion of tasks. However, \textbf{not all} processes benefit from multithreading. 

\item Resource sharing: \\

Threads share their recources amongst each other allowing for tasks to be completed in parallel in a
<<<<<<< HEAD
		single adress space. This is a great alternative to having separate processes and having \textit{inter-process communication (IPC)}
		which is often slower than thread shared memory.
		\footnote{Threads, from: Uic \today \\ \url{https://www.cs.uic.edu/~jbell/CourseNotes/OperatingSystems/4_Threads.html} }
=======
single adress space. \footnote{Threads, from: Uic \today \\ https://www.cs.uic.edu/~jbell/CourseNotes/OperatingSystems/4\_Threads.html }
>>>>>>> 9b37b3399fe78f00180157655b6c875cbcf369ed

\end{itemize}

\subsection{Program}

A program is a file containing a set of instructions. These instructions can be written in many different 
programming languages such as: Python, C or C++. These are considered high level programming languages as 
they have strong abstraction from the inner machinations of a computer. The CPU however cannot read these 
instructions yet, they first need to be translated into something that the computer can use. This is done 
via a compiler. The compiler translates a high level language to instructions and data. When the program is
loaded into memory it becomes a process and the instructions and data in memory can be seperated into four parts:

\begin{itemize}
\item The Stack: \\

The stack contains primarly local variables but also data such as function parameters and function return adresses. The data on the stack is often just used temporarily.
\item The Heap: \\

The heap is a memory area that is dynamically allocated at run time of the process.
\item The Data: \\

	The data is the section containing the initialized static and global variables. Strings such as \textt{"Hello, World!"} are stored
	in the data section.
\item The Text: \\

	The text segment contains CPU instructions for the process. The program counter (PC) points to this section because the instructions are
		fetched from the position in memory which the PC points to. Immediate values are also stored in this section.
		In instructions such as \texttt{mov \textless register\textgreater, 5} the immediate value is 5.
\end{itemize}

Further iformation on these four sections can be found within the memory chapter. \footnote{Processes, from: Tutorialspoint, \today \\ \url{https://www.tutorialspoint.com/operating_system/os_processes.htm} }


\subsection{Daemons and Init}

Daemons are computer programs that run in the background of an operating system. Regular users (i.e. users without \textit{Administrator rights} a.k.a \textit{root privileges} on unix-derivatives)
do not have direct access to daemon processes and have no control over them. The implementation for daemons will differ between
operating system platforms. For Microsoft Windows NT systems, the programs that serve the same functions as 
unix daemons are called \textit{Windows services}. In most cases they do not interact with user input and are started 
during the boot directly by the kernel.
Daemons provide functionalities such interprocess communication, hardware management or logging. 
Since Windows 2000, the services can be manually started and stopped via the control panel\footnote{Services, from: Microsoft, \today \\ \url{https://docs.microsoft.com/en-us/previous-versions/windows/it-pro/windows-server-2003/cc783643(v=ws.10)?redirectedfrom=MSDN}}.
In unix-like systems, daemon processes usually end with the letter "d" for example the crond daemon is 
a job scheduler for background processes. In Unix systems, there is also a special daemon from which all
other daemons spawn, namely the init daemon. The init daemon is the first process to be started at system boot 
and depending on configuration places the system in single user mode or spawns a login shell for other users. \footnote{Verma: Unix, (2006) P. 84 \\ \url{https://books.google.ch/books?id=JhS-TkW0tOYC&pg=PA84&redir_esc=y#v=onepage&q&f=false}}








%stuff:
%[^proc1]: https://medium.com/@imdadahad/a-quick-introduction-to-processes-in-computer-science-271f01c780da
%[^proc2]: https://www.cs.uic.edu/~jbell/CourseNotes/OperatingSystems/4_Threads.html 
%[^proc3]: https://www.tutorialspoint.com/operating_system/os_processes.htm
%[^proc4]: https://docs.microsoft.com/en-us/previous-versions/windows/it-pro/windows-server-2003/cc783643(v=ws.10)?redirectedfrom=MSDN
%[^proc5]: https://books.google.ch/books?id=JhS-TkW0tOYC&pg=PA84&redir_esc=y#v=onepage&q&f=false












